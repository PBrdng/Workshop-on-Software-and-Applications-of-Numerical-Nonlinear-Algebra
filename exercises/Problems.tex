\documentclass[11pt,reqno]{amsart}

% packages


\usepackage[top=3cm,bottom=3cm,left=3cm,right=3cm]{geometry}

% THEOREM Environments ---------------------------------------------------

\newtheorem{theorem}{Theorem}[section]
 \newtheorem{corollary}[theorem]{Corollary}
 \newtheorem{lemma}[theorem]{Lemma}
 \newtheorem{proposition}[theorem]{Proposition}
 \newtheorem{conjecture}{Conjecture}
  \newtheorem{question}{\sc Question}

  \theoremstyle{definition}
 \newtheorem{definition}[theorem]{Definition}
 \newtheorem{example}[theorem]{Example}

 \theoremstyle{remark}
 \newtheorem{remark}[theorem]{Remark}
 
 
\numberwithin{equation}{section}

\newcommand{\C}{\mathbb{C}}
\renewcommand{\t}{{\bf t}}
\newcommand{\fulges}[1]{{\color{cyan} #1}}
%\renewcommand{\St}{\mathfrak{St}}
\usepackage{diagbox}


\title{Problems and Exercises}

\author{P. Breiding, T. Brysiewicz, S. Telen, S. Timme}


\begin{document}
\maketitle

\section{Monodromy}
Let $F_c(x)$ be a zero-dimensional parametrized polynomial system with variables $x_1,\ldots,x_n$ and parameters $c_1,\ldots,c_k$. Let $Z \xrightarrow{\pi} \C^k$ be the branched cover where $Z=\{(x,p) | F_p(x)=0\}$ and $\pi:Z \to \C^k$ is the projection onto the parameters. Let $d$ be the degree of this branched cover.
Let $U$ be the set of regular values of $\pi$ and $G_\pi$ the monodromy group based at some point $p \in U$. 

\begin{enumerate}
\item Show $G_\pi$ is a group.

\item Show $G_\pi$ doesn't depend on the choice of $p \in U$ where you base monodromy loops. 


\item Show $G_\pi$ is transitive if and only if $Z$ has a unique irreducible component of maximal dimension.

\item Explain why $G_\pi$ being transitive is exactly the condition which allows \texttt{monodromy solve} to find all solutions to $\pi^{-1}(p)$.

\item Suppose $G_\pi$ is transitive. Show that $G_\pi$ is $2$-transitive if and only if the variety 
$$\{(x_1,x_2,p) \mid x_1,x_2 \in \pi^{-1}(p), p \in \C^k \}$$
has two maximal dimensional irreducible components 

\item Suppose $F_c(x)$ is defined over the real numbers.
 Is it possible for a real path in $U$ to produce a nontrivial monodromy permutation? Under which conditions can this happen?
 
\item As explained in previous lectures, solving a system $G(x)=0$ using homotopy methods requires one to embed $G(x)$ into a family of polynomial systems $F_c(x)$. Does the ability to solve $G(x)$ using monodromy depend on which family is chosen? 


\end{enumerate}

\section{Witness Sets}
\begin{enumerate}
\item Prove the trace test for plane curves. 

Hint: for one direction, it is useful to know that the monodromy group $$\{(x,L) \mid x \in X \cap L\} \xrightarrow{\pi} \textrm{Gr(2,3)}$$ $$(x,L) \mapsto L$$ is the full symmetric group

\item Compute a witness set for $$\text{SO}(5) = \{M \in \text{Mat}_{\mathbb{C}}(5,5) \mid M M^T = \textrm{id}, \det(M) = 1\}$$

\item How many maximal dimensional irreducible components does
$$\text{HSO}(4) = \{M \in \text{Mat}_{\mathbb{C}}(4,4) \mid MM^T= \textrm{id}, \det(M) = 1, M_{i,i}=0 \text{ for all }i\}$$
have? What are their degrees? How do they intersect?

\item The L\"uroth hypersurface $\mathfrak{L}$ is the hypersurface in the space of plane quartics parameterized by 
\begin{align*}
(\mathbb{C}^3)^5 &\to \mathbb{P}^{15}\\
(\ell_1,\ell_2,\ell_3,\ell_4,\ell_5) &\mapsto \sum_{i=1}^5 \prod_{j \neq i} \ell_j \text{ where } \ell_i = a_ix+b_iy+c_i
\end{align*}
Compute a witness set for $\mathfrak{L}$. What is its degree?


\end{enumerate}

\section{Other}

\begin{enumerate}
\item A conic is the zero set of a quadratic polynomial
$$
c(x,y) = a_1 x^2 + a_2 x y + a_3 y^2 + a_4 x + a_5 y + a_6
$$
with $a_i \in \mathbb{C}$.

[Emiris and Tzoumas](http://www.win.tue.nl/EWCG2005/Proceedings/38.pdf) write that there are 184 complex circles that are tangent to 3 general conics $C_1$, $C_2$ and $C_3$. This means, that there are 184 complex solutions $(a_1,a_2,r)$ such that there exists some $(x,y)\in\mathbb{C}^2$ with

$\bullet$ $(x-a_1)^2 + (y-a_2)^2 = r$,

$\bullet$ $(x,y)\in C_i$ for $1\leq i\leq 3$ and

$\bullet$ $(x-a_1, y-a_2)$ spans the normal space of $C_i$ at $(x,y)$ for $1\leq i\leq 3$.
\begin{enumerate}
\item Setup the polynomial system for 3 general conics and verify that this system has indeed 184 solutions

\item
Consider the three conics

$$C_1 = \{y=-x^2+2x+5\}, C_2 = \{y = 2x^2+5x-8\}$$
$$ \text{ and } C_3 = \{y = 8x^2-3x-2\}.$$

How many circles are tangent to these 3 conics? How many of them are real?

\item Find a configuration of 3 conics with as many real solutions as possible. It is possible to find 184 real solutions?

\end{enumerate}

\item A real algebraic variety is the common zero set of polynomials $f_1, \ldots, f_m \in \mathbb{R}[x_1,\ldots,x_n]$ denoted by $X=V(f_1,\ldots,f_m)$.

A bottleneck of $X$ is defined to be a pair of distinct points $x, y \in X$ such that $x-y$ is orthogonal to the tangent space $\mathrm{T}_x X$ and to $\mathrm{T}_y X$.

In [DEW18](https://arxiv.org/abs/1904.04502) it is shown that a generic plane curve of degree $d$ has $d^4 −5d^2 +4d$ bottleneck pairs. This is called the {bottleneck degree} of the curve.


Consider the curve $X=V(f)$ defined by $f=(x^4 + y^4 - 1)(x^2 + y^2 - 2) + x^5y$.
\begin{enumerate}
\item Write down definining equations for computing all bottlenecks.

\item What is the Bottleneck degree of $X$? How many real bottlenecks does it have?

\item What are the coordinates smallest bottleneck pair?

\item What effect do different start systems have on the number of paths necessary to track?

\item Visualize all bottlenecks for your favorite plane curve
\end{enumerate}

\item Consider a general quartic surface $f \in \mathbb{C}[x,y,z]$. 

This is defined by a random polynomial $f\in \mathbb{C}[x,y,z]$ of degree 4.

We want to count the number of planes in three-space which are tangent to $f$ in at least 3 points.
\begin{enumerate}
\item Set up polynomial systems to compute all tritangent planes of a general quartic surface. (Hint you should obtain a polynomial system in 11 variables).

\item What is the Bezout bound of the system in (a)?

\item Use the monodromy method to solve the system from (a).
\end{enumerate}

\end{enumerate}



\end{document}
